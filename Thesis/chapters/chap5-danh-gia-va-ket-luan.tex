\documentclass[../Thesis.tex]{subfiles}
\begin{document}
Trong đồ án này, tôi đã xây dựng thành công một hệ thống quản lý học tập đáng tin cậy và tiện ích. Qua quá trình nghiên cứu, phân tích yêu cầu và triển khai, tôi đã đạt được nhiều kết quả đáng chú ý.

Đầu tiên, tôi đã thiết kế và triển khai giao diện người dùng thân thiện và trực quan, giúp người dùng dễ dàng tương tác với hệ thống. Bằng cách sử dụng React JS, HTML, CSS và các công nghệ phát triển web khác, tôi đã tạo ra một giao diện đẹp mắt và linh hoạt.

Tiếp theo, tôi đã phát triển các tính năng quan trọng như quản lý khóa học, quản lý người dùng, thảo luận trực tuyến, gửi thông báo và chia sẻ tài liệu. Các tính năng này cho phép giảng viên và sinh viên tương tác và hỗ trợ quá trình học tập một cách hiệu quả.

Hệ thống cũng đã được triển khai trên nền tảng Google Cloud Server với sự hỗ trợ của Open Source Canvas LMS viết bằng Ruby. Sử dụng Google Cloud Server, tôi đã đảm bảo tính bảo mật, tốc độ và sẵn sàng của hệ thống. Đồng thời, việc sử dụng PostgreSQL và Mailgun đã cung cấp cho hệ thống tính năng mạnh mẽ và đáng tin cậy.

Tuy nhiên, trong quá trình xây dựng hệ thống, tôi nhận thấy còn một số thiếu sót và hướng phát triển tiềm năng trong tương lai. Các thiếu sót này bao gồm mở rộng tính năng, tối ưu hóa hiệu suất, nâng cao bảo mật, tương thích đa nền tảng và tích hợp công cụ hỗ trợ. Để cải thiện và nâng cao hệ thống, tôi đề xuất một số hướng phát triển sau:

    \begin{itemize}
        \item Mở rộng tính năng: Tôi có thể mở rộng tính năng của hệ thống bằng cách thêm các chức năng mới như diễn đàn trực tuyến, phân nhóm học tập và tích hợp công cụ tương tác trực tiếp.

        \item Tối ưu hóa hiệu suất: Để đảm bảo hệ thống hoạt động mượt mà và nhanh chóng, tôi có thể tối ưu hóa mã nguồn, cải thiện cơ sở dữ liệu và sử dụng các kỹ thuật tối ưu khác nhau.
    
        \item Nâng cao bảo mật: Để bảo vệ thông tin cá nhân và dữ liệu học tập, tôi có thể áp dụng các biện pháp bảo mật mạnh hơn như xác thực hai yếu tố, mã hóa dữ liệu và kiểm soát quyền truy cập.
    
        \item Tương thích đa nền tảng: Với sự phát triển của các thiết bị di động, tôi có thể nâng cấp giao diện và tính năng để tương thích trên nhiều nền tảng, bao gồm cả điện thoại di động và máy tính bảng.
    
        \item Tích hợp công cụ hỗ trợ: Tôi có thể tích hợp các công cụ hỗ trợ khác như trình quản lý tài liệu, trình duyệt đồ thị và công cụ theo dõi tiến độ học tập để tăng cường trải nghiệm người dùng.
    
    \end{itemize}
Tổng quan, dự án đã đạt được những kết quả đáng chú ý trong việc xây dựng hệ thống quản lý học tập. Mặc dù còn một số thiếu sót và hướng phát triển trong tương lai, tôi tự tin rằng với sự nỗ lực và cải tiến liên tục, hệ thống sẽ đáp ứng được nhu cầu của người dùng và đóng góp vào quá trình học tập hiệu quả.
\end{document}
