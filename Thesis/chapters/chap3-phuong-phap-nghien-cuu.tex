\documentclass[../Thesis.tex]{subfiles}
\begin{document}
\section{Thiết kế nghiên cứu}
    \subsection{Mục tiêu nghiên cứu}
    Mục tiêu chính của nghiên cứu này là phát triển một hệ thống quản lý học tập tùy chỉnh dựa trên Canvas LMS để phục vụ cho các tổ chức giáo dục trước đại học ở Việt Nam. Để đạt được mục tiêu này, chúng ta sẽ tập trung vào việc thiết kế giao diện người dùng hấp dẫn, dễ sử dụng và phù hợp với đặc điểm địa phương và văn hóa của Việt Nam. Ngoài ra, chúng ta cũng sẽ xem xét các yếu tố quan trọng khác như trải nghiệm người dùng và tương tác giữa giảng viên và sinh viên trong môi trường học tập trực tuyến.
    \subsection{Phương pháp nghiên cứu}
    Trong quá trình nghiên cứu, chúng ta sẽ sử dụng phương pháp nghiên cứu kết hợp giữa phương pháp định tính và phương pháp định lượng. Đầu tiên, chúng ta sẽ tiến hành phỏng vấn và cuộc trao đổi với các chuyên gia giáo dục, giảng viên và sinh viên để hiểu rõ hơn về nhu cầu và yêu cầu của các tổ chức giáo dục trước đại học ở Việt Nam. Những thông tin thu thập từ cuộc phỏng vấn sẽ giúp xác định các yếu tố cần thiết trong thiết kế giao diện người dùng và các tính năng cần có trong hệ thống LMS.

    Tiếp theo, chúng ta sẽ tiến hành một cuộc khảo sát trực tuyến để thu thập ý kiến từ cộng đồng giảng viên và sinh viên về sự hài lòng và khó khăn khi sử dụng các hệ thống LMS hiện có. Kết quả từ cuộc khảo sát sẽ giúp định hình và đánh giá các vấn đề cần giải quyết trong quá trình phát triển hệ thống quản lý học tập tùy chỉnh.

    Ngoài ra, chúng ta cũng sẽ tiến hành thử nghiệm thực tế với một số tổ chức giáo dục trước đại học ở Việt Nam để kiểm tra tính khả dụng và hiệu quả của hệ thống LMS được phát triển. Các cuộc thử nghiệm này sẽ cho phép chúng ta thu thập phản hồi từ giảng viên và sinh viên, từ đó đánh giá và cải thiện hệ thống theo hướng tốt nhất.
    \subsection{Phạm vi nghiên cứu}
    Phạm vi của nghiên cứu sẽ tập trung vào việc phát triển một hệ thống quản lý học tập tùy chỉnh dựa trên Canvas LMS cho các tổ chức giáo dục trước đại học ở Việt Nam. Chúng ta sẽ tìm hiểu và phân tích các yêu cầu và nhu cầu của các tổ chức giáo dục này, từ đó xây dựng giao diện người dùng phù hợp và tích hợp các tính năng cần thiết để nâng cao trải nghiệm học tập trực tuyến. Nghiên cứu cũng sẽ xem xét các yếu tố địa phương và văn hóa trong thiết kế giao diện người dùng, nhằm đảm bảo tính tương thích và thân thiện với người dùng tại Việt Nam.

    Hạn chế về thời gian và nguồn lực sẽ hạn chế phạm vi nghiên cứu chỉ trong phạm vi các tổ chức giáo dục trước đại học ở Việt Nam. Tuy nhiên, các kết quả và phương pháp nghiên cứu có thể được áp dụng và mở rộng sang các tổ chức giáo dục khác và các quốc gia khác trong tương lai.

    Trên cơ sở phạm vi nghiên cứu được xác định, chúng ta sẽ tiến hành các bước nghiên cứu chi tiết như đã trình bày trong các mục sau để đạt được mục tiêu nghiên cứu và giải quyết các vấn đề đã đề ra.
\section{Phương pháp thu thập dữ liệu}
Trong phần này, chúng tôi mô tả phương pháp thu thập dữ liệu được sử dụng trong nghiên cứu của chúng tôi để thu thập thông tin liên quan đến việc phát triển hệ thống quản lý học tập tùy chỉnh dựa trên Canvas LMS cho các tổ chức giáo dục trước đại học ở Việt Nam.
    \subsection{Phương pháp nghiên cứu đánh giá}
    Để đạt được mục tiêu nghiên cứu, chúng tôi áp dụng phương pháp nghiên cứu đánh giá để thu thập thông tin cần thiết. Điều này bao gồm việc xây dựng một mô hình đánh giá toàn diện để đo lường hiệu quả của hệ thống quản lý học tập tùy chỉnh dựa trên Canvas LMS trong việc cải thiện quá trình giảng dạy và học tập tại các tổ chức giáo dục trước đại học.
    \subsection{Phương pháp khảo sát}
    Chúng tôi cũng sử dụng phương pháp khảo sát để thu thập dữ liệu từ các giảng viên và sinh viên trong các tổ chức giáo dục trước đại học. Chúng tôi thiết kế một bộ câu hỏi khảo sát để đánh giá ý kiến, quan điểm và trải nghiệm của họ về việc sử dụng hệ thống quản lý học tập tùy chỉnh dựa trên Canvas LMS. Các câu hỏi trong khảo sát được xây dựng một cách cân nhắc để thu thập thông tin chi tiết về các yếu tố như tương tác người dùng, giao diện người dùng, tính năng và chức năng, hiệu quả giảng dạy và học tập, và sự tương thích với nhu cầu của các tổ chức giáo dục.
\section{Phương pháp phân tích dữ liệu}
Trong phần này, chúng tôi mô tả phương pháp phân tích dữ liệu được áp dụng trong nghiên cứu của chúng tôi để xử lý và hiểu thông tin thu thập từ các phương pháp thu thập dữ liệu đã được đề cập ở phần trước.
    \subsection{Phân tích dữ liệu định lượng}
    Chúng tôi sử dụng phương pháp phân tích dữ liệu định lượng để xử lý các dữ liệu thu thập từ khảo sát. Sau khi thu thập được câu trả lời từ các khảo sát, chúng tôi sẽ sắp xếp và tổ chức dữ liệu một cách cẩn thận. Tiếp theo, chúng tôi sẽ sử dụng các phương pháp thống kê và phân tích số liệu như phân tích tần số, phân tích tương quan và phân tích đa biến để hiểu rõ hơn về quan điểm, ý kiến và trải nghiệm của giảng viên và sinh viên về hệ thống quản lý học tập tùy chỉnh dựa trên Canvas LMS.
    \subsection{Phân tích nội dung}
    Đối với phương pháp thu thập dữ liệu từ tài liệu và các nguồn thông tin trực tuyến, chúng tôi sử dụng phương pháp phân tích nội dung. Chúng tôi tiến hành đọc và nghiên cứu các tài liệu liên quan, bài viết, sách và các nguồn thông tin khác để xác định các yếu tố quan trọng, tiêu chuẩn và quy trình liên quan đến việc phát triển hệ thống quản lý học tập tùy chỉnh dựa trên Canvas LMS. Chúng tôi rút ra các khái niệm, xu hướng và phương pháp từ tài liệu để đánh giá và so sánh với kết quả của nghiên cứu của chúng tôi.
    \subsection{Phân tích chất lượng dữ liệu}
    Cuối cùng, chúng tôi cũng thực hiện phân tích chất lượng dữ liệu để đảm bảo tính tin cậy và khả năng diễn giải của kết quả nghiên cứu. Chúng tôi kiểm tra các yếu tố như độ tin cậy của dữ liệu, độ đo và độ chính xác của kết quả thu thập để đảm bảo rằng dữ liệu thu được là đáng tin cậy và có thể sử dụng để đưa ra những
\section{Thiết kế lấy người dùng làm trung tâm}
Trong phần này, chúng tôi trình bày về tiếp cận thiết kế tập trung người dùng (User-Centered Design - UCD) được áp dụng trong quá trình phát triển hệ thống quản lý học tập tùy chỉnh dựa trên Canvas LMS. UCD là một phương pháp thiết kế dựa trên nghiên cứu và hiểu rõ nhu cầu, mong muốn, và hành vi của người dùng cuối để tạo ra một giao diện người dùng thân thiện, dễ sử dụng và mang lại trải nghiệm tốt cho người dùng.

Đầu tiên, chúng tôi tiến hành việc thu thập thông tin về người dùng tiềm năng của hệ thống quản lý học tập trước đại học. Chúng tôi tiến hành các cuộc phỏng vấn, khảo sát và quan sát người dùng để hiểu rõ nhu cầu, yêu cầu và thói quen của họ trong việc sử dụng hệ thống quản lý học tập.

Tiếp theo, chúng tôi thực hiện quá trình phân tích và xử lý dữ liệu thu thập từ người dùng. Chúng tôi sử dụng các phương pháp như phân tích người dùng, phân tích tác vụ và phân tích yêu cầu để định rõ các chức năng và tính năng cần có trong giao diện người dùng của hệ thống.

Dựa trên thông tin thu thập được và các phân tích trên, chúng tôi tiến hành quá trình thiết kế giao diện người dùng. Chúng tôi sử dụng các phương pháp như thiết kế dựa trên nguyên tắc, thiết kế người dùng và kiểm định để tạo ra các bản thiết kế giao diện người dùng sáng tạo và phù hợp với nhu cầu và mong muốn của người dùng.

Sau đó, chúng tôi thực hiện các vòng lặp thử nghiệm và đánh giá để kiểm tra và cải thiện giao diện người dùng. Chúng tôi tiến hành các cuộc thử nghiệm người dùng, thu thập phản hồi và tiến hành các điều chỉnh và cải tiến dựa trên phản hồi đó để đảm bảo rằng giao diện người dùng đáp ứng tốt nhu cầu và yêu cầu của người dùng.

Cuối cùng, chúng tôi tạo ra một thiết kế giao diện người dùng cuối cùng cho hệ thống quản lý học tập tùy chỉnh dựa trên Canvas LMS dựa trên quá trình thiết kế tập trung người dùng. Thiết kế này sẽ tối ưu hóa trải nghiệm người dùng, cung cấp tính tương thích, tùy chỉnh và dễ sử dụng cho các tổ chức giáo dục trước đại học ở Việt Nam.

\section{Phát triển nguyên mẫu}
Trong phần này, chúng tôi trình bày về quá trình phát triển nguyên mẫu của hệ thống quản lý học tập tùy chỉnh dựa trên Canvas LMS cho các tổ chức giáo dục trước đại học ở Việt Nam. Phát triển nguyên mẫu là một giai đoạn quan trọng trong quá trình thiết kế và phát triển hệ thống, nó giúp chúng tôi kiểm tra và đánh giá hiệu quả của các tính năng và chức năng được đề xuất.

Đầu tiên, chúng tôi xây dựng một nguyên mẫu tương tác (interactive prototype) dựa trên thiết kế giao diện người dùng đã hoàn thiện từ giai đoạn trước. Nguyên mẫu tương tác cho phép người dùng tương tác với giao diện và trải nghiệm các tính năng và chức năng của hệ thống trong môi trường ảo.

Tiếp theo, chúng tôi tiến hành quá trình thử nghiệm nguyên mẫu với người dùng thực tế. Chúng tôi tổ chức các buổi thảo luận, phỏng vấn và thu thập phản hồi từ người dùng để đánh giá sự tương tác, trải nghiệm và hiệu quả của nguyên mẫu.

Dựa trên phản hồi từ người dùng, chúng tôi tiến hành điều chỉnh và cải tiến nguyên mẫu, bao gồm cả giao diện người dùng và các tính năng, để đảm bảo rằng hệ thống đáp ứng tốt nhu cầu và mong muốn của người dùng.

Qua quá trình phát triển nguyên mẫu, chúng tôi nhận được phản hồi quan trọng từ người dùng và cải thiện đáng kể giao diện người dùng và trải nghiệm người dùng của hệ thống quản lý học tập. Nguyên mẫu tương tác cung cấp một cái nhìn trực quan về cách hệ thống sẽ hoạt động và cho phép chúng tôi thực hiện các điều chỉnh cuối cùng trước khi tiến hành giai đoạn triển khai và triển khai thực tế.
\end{document}
